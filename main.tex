\documentclass{umonsreport}
\usepackage[french]{babel} %Langue
\usepackage{lipsum}

\begin{document}

%----------- Informations ---------

\title{Umons Report - Template}
\fulltitle{Titre du rapport} %Gros titre en gras
\smalltitle{R\'esum\'e du titre} %Apparait en bas de page

\class{Le cours}
\prof{Pr\'enom \textsc{Nom}} %Professeur
\assistant{Pr\'enom \textsc{Nom}} %Assistant(s) /!\ MODIFIER LE PLURIEL SUR LA PAGE DE GARDE SI PLUSIEURS 
\auth{Pr\'enom \textsc{Nom} {\normalsize\texttt{(200200)}} } %Auteur(s) /!\ MODIFIER LE PLURIEL SUR LA PAGE DE GARDE SI PLUSIEURS 


%------------------- Init -------------------
        
\begin{titlepage}
    \tikz[remember picture,overlay] \node[opacity=.5,inner sep=0pt] at (current page.center){\includegraphics[height=\paperheight]{logos/background.png}};
    \centering
    
        \includegraphics[height=2.1cm]{logos/Umons.png}
        \hfill
        \includegraphics[height=2.3cm]{logos/FS-grey.png}\\
        \vspace{3cm}
        
    \HRule\vspace*{.5cm}
        {\huge\bfseries\fulltitle}\vspace*{.15cm} 
    \HRule\vspace*{2.5cm}
    
    \begin{minipage}[t]{0.4\textwidth}
        \begin{flushleft} \large
            \textbf{Auteur}\\
            \auth
        \end{flushleft}
    \end{minipage}
    ~
    \begin{minipage}[t]{0.4\textwidth}
        \begin{flushright} \large
            \textbf{Professeur} \\
            \prof
            %\\\vspace*{.5cm}
            %\textbf{Assistant} \\
            %\assistant
        \end{flushright}
    \end{minipage}
    
    %If you dont want to include professor, uncomment the two following lines and delete both minipages
    %\Large \textbf{Auteur:}\\
    %\auth
    
    \vfill
    %\today
    Année académique 2023-2024
    
    %\vspace*{1.5cm}
    %\includegraphics[height=2.1cm]{logos/Umons.png}
    %\hfill
    %\includegraphics[height=2.3cm]{logos/FS-info2.png}
\end{titlepage} %IMPORTEZ LA VOTRE SI VOUS PREFEREZ
\margin %En-tetes et bas de pages
\tableofcontents
\newpage

%------------ Body ----------------


\section{Section uno} 

\lipsum[3-4]

\subsection{Sous-section uno}

\lipsum[3-4]

\section{Section dos}

\lipsum[3-5]

\section{Amazing logos !}

\begin{figure}[H]
    \centering
    \begin{subfigure}[t]{.4\textwidth}
        \centering
        \includegraphics[height=7em]{logos/FS-small.png}
    \end{subfigure}
    ~
    \begin{subfigure}[t]{.4\textwidth}
        \centering
        \includegraphics[width=\textwidth]{logos/FS-grey-red-box.png}
    \end{subfigure}
    ~ \bigskip
    \begin{subfigure}[t]{.4\textwidth}
        \centering
        \includegraphics[width=\textwidth]{logos/FS.png}
    \end{subfigure}
    ~ 
    \begin{subfigure}[t]{.4\textwidth}
        \centering
        \includegraphics[width=\textwidth]{logos/FS-grey.png}
    \end{subfigure}
    ~ 
    \begin{subfigure}[t]{.4\textwidth}
        \centering
        \includegraphics[width=\textwidth]{logos/FS-info.png}
    \end{subfigure}
    ~ 
    \begin{subfigure}[t]{.4\textwidth}
        \centering
        \includegraphics[width=\textwidth]{logos/FS-info-grey-EN.png}
    \end{subfigure}
    \caption{Alternatives au logo au bas à gauche de la page de garde}
    \label{fs-logos}
\end{figure}

Beaucoup de ces logos ont une version anglophone, essayez le même nom avec un \texttt{"-EN.png"} à la fin. Voici en exemple en figure \ref{en-logos}.
\begin{figure}[H]
    \centering
    \begin{subfigure}[t]{.4\textwidth}
        \centering
        \includegraphics[width=\textwidth]{logos/Umons.png}
    \end{subfigure}
    ~\hspace{3em}
    \begin{subfigure}[t]{.4\textwidth}
        \centering
        \includegraphics[width=\textwidth]{logos/Umons-EN.png}
    \end{subfigure}
    \caption{Exemple de version anglophone}
    \label{en-logos}
\end{figure}

Pour le logo en en-tête, il existe une version monochrome.
\begin{figure}[H]
    \centering
    \begin{subfigure}[t]{.4\textwidth}
        \centering
        \includegraphics[width=\textwidth]{logos/small-Umons.png}
    \end{subfigure}
    ~\hspace{3em}
    \begin{subfigure}[t]{.4\textwidth}
        \centering
        \includegraphics[width=\textwidth]{logos/small-Umons-black.png}
    \end{subfigure}
    \caption{Alternatives au logo en en-tête}
    \label{heading-logos}
\end{figure}

\begin{reminder}[p-value]
    The \textbf{\textit{p-value}} of a test is the smallest level for $\alpha$ that allows us to reject the \textit{hypothesis}.
    \begin{itemize} 
        \item If the choosen $\alpha$ is bigger or equal to the \textit{p-value}, we can \textbf{reject} the hypothesis. 
        \item Otherwise, we can \textbf{accept} it (without conviction).
    \end{itemize}
\end{reminder}

\begin{definition}[Théorème fondamental de l'analyse]
    Soit $f:[a,b]\to \mathbb R$ une application continue. Alors, l'application
    \[
    \mathcal{F}:[a,b]\to \mathbb R:x\mapsto\int_a^xf(\epsilon)d\epsilon
    \]
    est une \textbf{primitive} de $f$ sur $[a,b]$, c'est \`a dire que, pour tout $x\in[a,b]$,
    \[
    \partial\mathcal{F}(x)=f(x)
    \]
\end{definition}

Exemple d'algo avec \verb|algorithm2e|\\


\SetKwComment{Comment}{/* }{ */}
\begin{algorithm}[H]
\caption{An algorithm with caption}\label{alg:two}
\KwData{$n \geq 0$}
\KwResult{$y = x^n$}
$y \gets 1$\;
$X \gets x$\;
$N \gets n$\;
\While{$N \neq 0$}{
  \eIf{$N$ is even}{
    $X \gets X \times X$\;
    $N \gets \frac{N}{2} $ \Comment*[r]{This is a comment}
  }{\If{$N$ is odd}{
      $y \gets y \times X$\;
      $N \gets N - 1$\;
    }
  }
}
\end{algorithm}
Pour du pseudo code en français, ajouter les options \verb|french| et \verb|frenchkw|. Les mots-clés en français sont dispo à la page 42 de la \href{https://tug.ctan.org/macros/latex/contrib/algorithm2e/doc/algorithm2e.pdf}{\textit{doc du package}}

\end{document}